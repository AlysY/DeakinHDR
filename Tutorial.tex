% Options for packages loaded elsewhere
\PassOptionsToPackage{unicode}{hyperref}
\PassOptionsToPackage{hyphens}{url}
%
\documentclass[
]{article}
\usepackage{lmodern}
\usepackage{amsmath}
\usepackage{ifxetex,ifluatex}
\ifnum 0\ifxetex 1\fi\ifluatex 1\fi=0 % if pdftex
  \usepackage[T1]{fontenc}
  \usepackage[utf8]{inputenc}
  \usepackage{textcomp} % provide euro and other symbols
  \usepackage{amssymb}
\else % if luatex or xetex
  \usepackage{unicode-math}
  \defaultfontfeatures{Scale=MatchLowercase}
  \defaultfontfeatures[\rmfamily]{Ligatures=TeX,Scale=1}
\fi
% Use upquote if available, for straight quotes in verbatim environments
\IfFileExists{upquote.sty}{\usepackage{upquote}}{}
\IfFileExists{microtype.sty}{% use microtype if available
  \usepackage[]{microtype}
  \UseMicrotypeSet[protrusion]{basicmath} % disable protrusion for tt fonts
}{}
\makeatletter
\@ifundefined{KOMAClassName}{% if non-KOMA class
  \IfFileExists{parskip.sty}{%
    \usepackage{parskip}
  }{% else
    \setlength{\parindent}{0pt}
    \setlength{\parskip}{6pt plus 2pt minus 1pt}}
}{% if KOMA class
  \KOMAoptions{parskip=half}}
\makeatother
\usepackage{xcolor}
\IfFileExists{xurl.sty}{\usepackage{xurl}}{} % add URL line breaks if available
\IfFileExists{bookmark.sty}{\usepackage{bookmark}}{\usepackage{hyperref}}
\hypersetup{
  pdftitle={R Tips and Tricks},
  pdfauthor={Alys Young},
  hidelinks,
  pdfcreator={LaTeX via pandoc}}
\urlstyle{same} % disable monospaced font for URLs
\usepackage[margin=1in]{geometry}
\usepackage{color}
\usepackage{fancyvrb}
\newcommand{\VerbBar}{|}
\newcommand{\VERB}{\Verb[commandchars=\\\{\}]}
\DefineVerbatimEnvironment{Highlighting}{Verbatim}{commandchars=\\\{\}}
% Add ',fontsize=\small' for more characters per line
\usepackage{framed}
\definecolor{shadecolor}{RGB}{248,248,248}
\newenvironment{Shaded}{\begin{snugshade}}{\end{snugshade}}
\newcommand{\AlertTok}[1]{\textcolor[rgb]{0.94,0.16,0.16}{#1}}
\newcommand{\AnnotationTok}[1]{\textcolor[rgb]{0.56,0.35,0.01}{\textbf{\textit{#1}}}}
\newcommand{\AttributeTok}[1]{\textcolor[rgb]{0.77,0.63,0.00}{#1}}
\newcommand{\BaseNTok}[1]{\textcolor[rgb]{0.00,0.00,0.81}{#1}}
\newcommand{\BuiltInTok}[1]{#1}
\newcommand{\CharTok}[1]{\textcolor[rgb]{0.31,0.60,0.02}{#1}}
\newcommand{\CommentTok}[1]{\textcolor[rgb]{0.56,0.35,0.01}{\textit{#1}}}
\newcommand{\CommentVarTok}[1]{\textcolor[rgb]{0.56,0.35,0.01}{\textbf{\textit{#1}}}}
\newcommand{\ConstantTok}[1]{\textcolor[rgb]{0.00,0.00,0.00}{#1}}
\newcommand{\ControlFlowTok}[1]{\textcolor[rgb]{0.13,0.29,0.53}{\textbf{#1}}}
\newcommand{\DataTypeTok}[1]{\textcolor[rgb]{0.13,0.29,0.53}{#1}}
\newcommand{\DecValTok}[1]{\textcolor[rgb]{0.00,0.00,0.81}{#1}}
\newcommand{\DocumentationTok}[1]{\textcolor[rgb]{0.56,0.35,0.01}{\textbf{\textit{#1}}}}
\newcommand{\ErrorTok}[1]{\textcolor[rgb]{0.64,0.00,0.00}{\textbf{#1}}}
\newcommand{\ExtensionTok}[1]{#1}
\newcommand{\FloatTok}[1]{\textcolor[rgb]{0.00,0.00,0.81}{#1}}
\newcommand{\FunctionTok}[1]{\textcolor[rgb]{0.00,0.00,0.00}{#1}}
\newcommand{\ImportTok}[1]{#1}
\newcommand{\InformationTok}[1]{\textcolor[rgb]{0.56,0.35,0.01}{\textbf{\textit{#1}}}}
\newcommand{\KeywordTok}[1]{\textcolor[rgb]{0.13,0.29,0.53}{\textbf{#1}}}
\newcommand{\NormalTok}[1]{#1}
\newcommand{\OperatorTok}[1]{\textcolor[rgb]{0.81,0.36,0.00}{\textbf{#1}}}
\newcommand{\OtherTok}[1]{\textcolor[rgb]{0.56,0.35,0.01}{#1}}
\newcommand{\PreprocessorTok}[1]{\textcolor[rgb]{0.56,0.35,0.01}{\textit{#1}}}
\newcommand{\RegionMarkerTok}[1]{#1}
\newcommand{\SpecialCharTok}[1]{\textcolor[rgb]{0.00,0.00,0.00}{#1}}
\newcommand{\SpecialStringTok}[1]{\textcolor[rgb]{0.31,0.60,0.02}{#1}}
\newcommand{\StringTok}[1]{\textcolor[rgb]{0.31,0.60,0.02}{#1}}
\newcommand{\VariableTok}[1]{\textcolor[rgb]{0.00,0.00,0.00}{#1}}
\newcommand{\VerbatimStringTok}[1]{\textcolor[rgb]{0.31,0.60,0.02}{#1}}
\newcommand{\WarningTok}[1]{\textcolor[rgb]{0.56,0.35,0.01}{\textbf{\textit{#1}}}}
\usepackage{graphicx}
\makeatletter
\def\maxwidth{\ifdim\Gin@nat@width>\linewidth\linewidth\else\Gin@nat@width\fi}
\def\maxheight{\ifdim\Gin@nat@height>\textheight\textheight\else\Gin@nat@height\fi}
\makeatother
% Scale images if necessary, so that they will not overflow the page
% margins by default, and it is still possible to overwrite the defaults
% using explicit options in \includegraphics[width, height, ...]{}
\setkeys{Gin}{width=\maxwidth,height=\maxheight,keepaspectratio}
% Set default figure placement to htbp
\makeatletter
\def\fps@figure{htbp}
\makeatother
\setlength{\emergencystretch}{3em} % prevent overfull lines
\providecommand{\tightlist}{%
  \setlength{\itemsep}{0pt}\setlength{\parskip}{0pt}}
\setcounter{secnumdepth}{-\maxdimen} % remove section numbering
\ifluatex
  \usepackage{selnolig}  % disable illegal ligatures
\fi

\title{R Tips and Tricks}
\author{Alys Young}
\date{19/03/2021}

\begin{document}
\maketitle

This tutorial documents some of the tips and tricks that we Deakin HDR
students have learnt over the years of coding. The idea and some topics
were based on a QAEco coding club session at Uni Melb by Dr Saras
Windecker.

\hypertarget{setting-up-your-projects}{%
\subsection{Setting up your projects}\label{setting-up-your-projects}}

\hypertarget{rstudio-layout}{%
\subsubsection{RStudio layout}\label{rstudio-layout}}

You dont have to keep the layout of R studio the same as the way it
starts, find a layout, colour, font that works best for you. E.g. I love
having my scripts panel and console the biggest, and the environment
panel small (Tools \textgreater{} Global Options \textgreater{} Pane
layout). Lots of people like using dark modes for the colour (Tools
\textgreater{} Global Options \textgreater{} Appearance). You can also
use rainbow parenthesis to help you get the right number of brackets
(Tools \textgreater{} Global Options \textgreater{} Code \textgreater{}
Editing). Play around with these in tools \textgreater{} global options.

\hypertarget{r-projects}{%
\subsubsection{R Projects}\label{r-projects}}

Using R projects rather than just scripts helps to organise all the
projects that you work on. Projects create their own directory for that
project and this is set as the working directory, keeps the environments
of different projects seperate, allow you to open multiple projects at
once and run multiple things at once, and are used for version control.
To create a new project, go to File \textgreater{} New Project

\hypertarget{folders}{%
\subsubsection{Folders}\label{folders}}

Everyone sets up their folders different, so find a way that works for
you Here are some examples: * Data\_raw - where all the raw data goes\\
* Data\_process - where all the data you are processing and cleaning
goes\\
* Data\_clean - final data used in your next steps\\
* Outputs - Your good outputs\\
* Scripts - the scripts you are using\\
* Archive\_scripts - Old scripts\\
* Archive\_outputs - old outputs

\hypertarget{scripts}{%
\subsubsection{Scripts}\label{scripts}}

It seems like personal preference if you like to have many small scripts
or fewer big ones. Most people like to have functions that you create in
seperate scripts which are sourced in (described below). If you like to
split you scripts by what their function is, it can be good to name them
using numbers to indicate the order they go in
e.g.~\texttt{1\_DataCleaning}, \texttt{2\_DataManipulation},
\texttt{3\_Models}, \texttt{4\_Plots}. Other people also have a master
scripts which contain all the essential code needed to reproduce the
results. This master script is good if you want to publish your code.

\hypertarget{r-markdown}{%
\subsubsection{R Markdown}\label{r-markdown}}

These notes are written in RMarkdown. Its a method of writing doccuments
and easily embedding code and code outputs. People use RMarkdown for
writting papers, communicating with stakeholders and supervisors,
creating resumes ect. My new favourite way to use it is like a tradition
lab notebook which records what we tried and why. I have started doing
the data cleaning and manipulation in markdown so that I have a record
of what we tried, why decisions were made, and communicate that process
to supervisors.

\hypertarget{setting-up-you-code}{%
\subsection{Setting up you code}\label{setting-up-you-code}}

Along with the heading at the top of the script, i also always include
\texttt{rm(list=ls())} at the top too. This clears all the code in your
consol. Sometimes you want to keep elements there if they take a long
time to create or run, but ideally you would still clean it so that you
know those elements can still be created from the code in your script.
Otherwise you might be using elements which dont have the value that you
expct which will change your results and make the code non-reproducible.

\hypertarget{load-packages}{%
\subsubsection{Load packages}\label{load-packages}}

If you're sharing your code, it can be difficult to know which packages
are already installed on someone elses computer. use this structure
\texttt{if\ (!require(ggplot2))\ install.packages(\textquotesingle{}ggplot2\textquotesingle{})},
changing \texttt{ggplot2} to be the package you want installed. You can
then load packages as usual using \texttt{library(ggplot2)}. But you can
also use packages without loading them by using \texttt{::} symbols
after the package name and then after the \texttt{::} write the function
you want to use. For example, if we wanted to manipulate data using the
dplyr package and the function mutate, we could say
\texttt{dplyr::mutate()}. This would be the same as loading the
\texttt{dplyr} packages and writing the function mutate
\texttt{mutate()}

\hypertarget{source-a-file}{%
\subsubsection{Source a file}\label{source-a-file}}

Write code in another file which makes elements in an environment and
use the function \texttt{source} to load the R script and all its ojects
into your environment. Try it by making a new blank scipt, code a simple
vector, e.g.~\texttt{x\ \textless{}-\ c(1,4,9)} and save the script. Go
to a new script, \texttt{source()} the script with the vector code in
it,
e.g.~\texttt{source(\textquotesingle{}your\_directory/your\_script.R\textquotesingle{})}
and \texttt{x} should appear in your environmnet.

\hypertarget{namimg-of-elements}{%
\subsubsection{Namimg of elements}\label{namimg-of-elements}}

It can be easier to understand your code if elements are named in a
logical manner. For example, shapefiles could have the suffix
\texttt{\_shp}, dataframes \texttt{\_df}, rasters \texttt{\_ras}.
Remember, its important that code can be read by humans so while being
succient is good, it can be better to write longer code that is easier
to understand by yourself and other people.

\hypertarget{headings-and-commenting}{%
\subsection{Headings and commenting}\label{headings-and-commenting}}

Using logical heading and commenting out your code are useful in
organisation and help if you ever have to come back to the code later.
This section shows you how Alys Young lays out her code to give you
inpiration. I find boxes around heading easy to find in the script, but
I use starts * (shift+8) instead of hashes \# (shift+3) the whole way to
keep my automatic table of contents cleaner. The table of contents can
be accessed for each script by clicking the button on the top right of
an active script which looks like horizontal lines (in line with the run
button, next to the source button).

\hypertarget{top-of-the-doccument}{%
\subsubsection{Top of the doccument}\label{top-of-the-doccument}}

Write detailed notes with important information at the top of your
script.

\begin{Shaded}
\begin{Highlighting}[]
\DocumentationTok{\#\#\#**************}\AlertTok{\#\#\#}
\DocumentationTok{\#\#\# Script title }\AlertTok{\#\#\#}
\DocumentationTok{\#\#\#**************}\AlertTok{\#\#\#}
\CommentTok{\# Project aim:}
\CommentTok{\#}
\CommentTok{\# Author:}
\CommentTok{\#}
\CommentTok{\# Collaborators:}
\CommentTok{\#}
\CommentTok{\# Date:}
\CommentTok{\#}
\CommentTok{\# Script aim:}
\end{Highlighting}
\end{Shaded}

\hypertarget{in-the-code}{%
\subsubsection{In the code}\label{in-the-code}}

For main sections of the code I use boxes again and number them. My
first main section is usually ``set up''.

\begin{Shaded}
\begin{Highlighting}[]
\DocumentationTok{\#\#************************\#\#}
\DocumentationTok{\#\# 1. Write headings here \#\# {-}{-}{-}{-}{-}{-}{-}{-}{-}{-}{-}{-}{-}{-}{-}{-}{-}{-}{-}{-}{-}{-}{-}{-}{-}{-}{-}{-}{-}{-}{-}{-}{-}{-}{-}{-}{-}{-}{-}{-}{-}{-}{-}{-}{-}{-}{-}{-}{-}{-}{-}{-}{-}{-}{-}{-}{-}{-}{-}{-}{-}{-}{-}{-}{-}{-}{-}{-}{-}{-}{-}{-}{-}{-}{-}{-}{-}{-}{-}{-}{-}{-}{-}{-}{-}{-}{-}{-}{-}{-}{-}{-}{-}{-}{-}{-}{-}{-}{-}{-}{-}{-}{-}}
\DocumentationTok{\#\#************************\#\#}
\end{Highlighting}
\end{Shaded}

Sub headings can be more simple. By using the line after the title, it
will appear in the table of contents. Otherwise you can use 4 or more
hash, hyphen or lines. Hashes \#\#\#\# Hyphen ---- Equals ====

\begin{Shaded}
\begin{Highlighting}[]
\CommentTok{\# 1.1 Write sub{-}headings like this {-}{-}{-}{-}{-}{-}{-}{-}{-}{-}{-}{-}{-}{-}{-}{-}{-}{-}{-}{-}{-}{-}{-}{-}{-}{-}{-}{-}{-}{-}{-}{-}{-}{-}{-}{-}{-}{-}{-}{-}{-}{-}{-}{-}{-}{-}{-}{-}{-}{-}{-}{-}{-}{-}{-}{-}{-}{-}{-}{-}{-}}
\end{Highlighting}
\end{Shaded}

\hypertarget{snippets}{%
\subsubsection{Snippets}\label{snippets}}

You can create pre-made pieces of code so that you don't have to type it
out every time; these are called snippets. Snippets can be viewed and
modified in tools \textgreater{} global options \textgreater{} code
\textgreater{} edit snippets Here you can see what snippets are
available by defult and create your own using the same format as the
ones currently there.\\
For example, I use a snippet for the the heading for the top of the
document as shown below.

\begin{Shaded}
\begin{Highlighting}[]
\NormalTok{snippet header}
    \DocumentationTok{\#\#\#***************}\AlertTok{\#\#\#}
    \DocumentationTok{\#\#\# Project Title }\AlertTok{\#\#\#}
    \DocumentationTok{\#\#\#***************}\AlertTok{\#\#\#}
    \CommentTok{\# Project aim:}
    \CommentTok{\#}
    \CommentTok{\# Author:}
    \CommentTok{\#}
    \CommentTok{\# Collaborators:}
    \CommentTok{\#}
    \CommentTok{\# Date:}
    \CommentTok{\#}
    \CommentTok{\# Script aim:}
\end{Highlighting}
\end{Shaded}

Thank you to August Hao for showing me snippets.

\hypertarget{other-notes}{%
\subsection{Other notes}\label{other-notes}}

\hypertarget{key-board-shortcuts}{%
\subsubsection{Key board shortcuts}\label{key-board-shortcuts}}

Shortcuts are a great way to increase your speed and make coding easier.
They can be found here
\url{https://support.rstudio.com/hc/en-us/articles/200711853-Keyboard-Shortcuts}
. Some of the ones I use most often: * A new heading\\
* Windows: ctrl + shift + r * Mac: cmnd + shift + r

\begin{itemize}
\tightlist
\item
  Dplyr pipe

  \begin{itemize}
  \tightlist
  \item
    Windows: ctrl + shift + m
  \item
    Mac: cmnd + shift + m
  \end{itemize}
\item
  To jump to the end of your line of code

  \begin{itemize}
  \tightlist
  \item
    Windows: crt + arrow key
  \item
    Mac: cmd + arrow
  \end{itemize}
\end{itemize}

\end{document}
